\documentclass{\string~/.hexagon/assets/statement}

\title{北京一零一中学信息学奥林匹克竞赛模拟赛}

\fancyhead[R]{\footnotesize \MakeLowercase{\fancyplain{}{\leftmark}}}
\renewcommand{\sectionmark}[1]{\markboth{#1}{}} 


\begin{document}
\begin{titlepage}
  \vspace*{-20mm}
  \begin{center}
    \LARGE \sffamily \thetitle \\
    \Huge {\rmfamily Div 1} 模拟赛\\
    % \LARGE {\rmfamily Div 1}
    % \includegraphics[height = 32 pt]{11.png} 特供版
    % \par\mdseries\LARGE\SubtitleContent\vskip 0.5em

    \large \rmfamily 时间:2024 年 5 月 25 日 13:00 \textasciitilde\ 16:00
    \vskip 0.5em
  \end{center}

  \begin{center}
    
\begin{tabularx}{\textwidth}{|p{3.2cm}|X|X|X|}
\hline
题目名称 & 中国剩余定理 & 中国剩余定理 & 中国剩余定理\\ \hline 题目类型 & 传统题 & 传统题 & 传统题\\ \hline 目录 & \texttt{crt} & \texttt{crt} & \texttt{crt}\\ \hline 可执行文件名 & \texttt{crt} & \texttt{crt} & \texttt{crt}\\ \hline 输入文件名 & \texttt{crt.in} & \texttt{crt.in} & \texttt{crt.in}\\ \hline 输出文件名 & \texttt{crt.out} & \texttt{crt.out} & \texttt{crt.out}\\ \hline 测试点时限 & 1秒 & 1秒 & 1秒\\ \hline 内存限制 & 512 MiB & 512 MiB & 512 MiB\\ \hline 测试点数目 & 1 & 1 & 1\\ \hline 测试点是否等分 & 是 & 是 & 是\\ \hline 
\end{tabularx}\par

  \end{center}

  提交源程序文件名
  \begin{center}
    
\begin{tabularx}{\textwidth}{|p{3.2cm}|X|X|X|}
\hline
对于 C++ & \texttt{crt.cpp} & \texttt{crt.cpp} & \texttt{crt.cpp} \\
\hline
\end{tabularx}\par

  \end{center}

  编译选项
  \begin{center}
    \begin{tabularx}{\textwidth}{|p{3.2cm}|X|X|X|}
      \hline
      对于 C++ & \multicolumn{3}{>{\centering\hsize=\dimexpr3\hsize+4\tabcolsep+2\arrayrulewidth\relax}X|}{\texttt{-O2 -std=c++14 -static}} \\
      \hline
    \end{tabularx}\par
  \end{center}

  \begingroup\titleformat{\subsection}{\bf}{}{0pt}{\hspace{0.5em}}\subsection{注意事项与提醒(请选手务必仔细阅读)}\endgroup
  \begin{enumerate}
    \item 选手提交的源程序必须存放在\stress{已建立}好的,且\stress{带有样例文件和下发文件的}的文件夹中,文件名称与对应试题英文名一致;
    \item 文件名(包括程序名和输入输出文件名)必须使用英文小写。
    \item C++ 中函数 main() 的返回值类型必须是 int,值必须为 0。
    \item \stress{对于因未遵守以上规则对成绩造成的影响,相关申诉不予受理}。
    \item 若无特殊说明,结果比较方式为\stress{忽略行末空格、文末回车后的全文比较。}。
    \item 程序可使用的栈空间大小与该题内存空间限制一致。
    \item 在终端中执行命令 \texttt{ulimit -s unlimited} 可将当前终端下的栈空间限制放大,但你使用的栈空间大小不应超过题目限制。
    \item 若无特殊说明,每道题的\stress{代码大小限制为 100KB}。
    \item 若无特殊说明,输入与输出中同一行的相邻整数、字符串等均使用一个空格分隔。
    \item 输入文件中可能存在行末空格,请选手使用更完善的读入方式(例如 scanf 函数)避免出错。
    \item 直接复制 PDF 题面中的多行样例,数据将带有行号,建议选手直接使用对应目录下的样例文件进行测试。
    \item 使用 std::deque 等 STL 容器时,请注意其内存空间消耗。
    \item 请务必使用题面中规定的的编译参数,保证你的程序在本机能够通过编译。此外\stress{不允许在程序中手动开启其他编译选项},一经发现,本题成绩以 0 分处理。
  \end{enumerate}
\end{titlepage}

\clearpage

% problem statement for crt

\renewcommand{\cnname}{\protect\input{crt/1-CN-NAME}\unskip}
\renewcommand{\enname}{\protect\input{crt/2-EN-NAME}\unskip}

\section{\cnname(\englishname{\enname})}

\subsection[题目描述]{【题目描述】}
「物不知数」问题:
\begin{quote}
  有物不知其数,三三数之剩二,五五数之剩三,七七数之剩二。问物几何?
\end{quote}
即求满足以下条件的整数:除以 \(3\) 余 \(2\),除以 \(5\) 余 \(3\),除以
\(7\) 余 \(2\)。该问题最早见于《孙子算经》中,并有该问题的具体解法。宋朝数学家秦九韶于
1247
年《数书九章》卷一、二《大衍类》对「物不知数」问题做出了完整系统的解答。上面具体问题的解答口诀由明朝数学家程大位在《算法统宗》中给出:
\begin{quote}
  三人同行七十希,五树梅花廿一支,七子团圆正半月,除百零五便得知。
\end{quote}
\(2\times 70+3\times 21+2\times 15=233=2\times 105+23\),故答案为
\(23\)。

用现代数学的语言来说明的话,中国剩余定理 (Chinese Remainder Theorem, CRT)
可求解如下形式的一元线性同余方程组(其中 \(n_1, n_2, \cdots, n_k\)
两两互质):

\[
  \left\{  \begin{aligned}
    x & \equiv a_1 \pmod {n_1} \\
    x & \equiv a_2 \pmod {n_2} \\
      & \vdots                 \\
    x & \equiv a_k \pmod {n_k} \\
  \end{aligned}\right.
\]
上面的「物不知数」问题就是一元线性同余方程组的一个实例。
使用中国剩余定理求解一元线性同余方程组的步骤如下:
\begin{enumerate}
  \item
        计算所有模数的积 \(n\);
  \item
        对于第 \(i\) 个方程:

        \begin{enumerate}
          \item
                计算 \(m_i=\frac{n}{n_i}\);
          \item
                计算 \(m_i\) 在模 \(n_i\) 意义下的逆元
                \(m_i^{-1}\);
          \item
                计算 \(c_i=m_im_i^{-1}\)(不要对 \(n_i\) 取模)。
        \end{enumerate}
  \item
        方程组在模 \(n\) 意义下的唯一解为:\(x=\sum_{i=1}^k a_ic_i \pmod n\)。
\end{enumerate}
中国剩余定理的一个应用是,如果要计算$a\bmod n$,其中$n=p_1^{c_1}p_2^{c_2}\cdots p_k^{c_k}$,那么可以先计算$a_i=a\bmod p_i^{c_i}$,然后使用中国剩余定理求解出$a\mod n$的值,这样就可以在$n$是一个比较大的合数的时候减小乘法运算的规模。

Alice想要计算两个数字$a,b$的乘积对某个整数$n$取模的结果。在学习了中国剩余定理后,她决定采用上述的方法,先对$n$进行质因数分解,然后计算答案对每个质因数取模的结果,最后使用中国剩余定理求解出答案。但是很悲惨的是,在计算过程中,对于其中一个质因数的运算过程出现了错误,导致最终的结果也是错误的。虽然经过重新计算她得出了正确的结果,但是她还是想知道她之前的计算过程中在哪个质因数上出错了。
\subsection[输入格式]{【输入格式】}
从文件 \filename{\enname.in} 中读入数据。
输入的第一行包含一个整数 $T$,表示数据组数。

接下来 $T$ 行,每行包含四个整数 $a,b,m,n$,分别表示要计算乘积的两个数,Alice计算出的错误结果,以及模数。

\subsection[输出格式]{【输出格式】}
输出到文件 \filename{\enname.in} 中。
输出到文件 \filename{\enname.out} 中。

输出$T$行,每行一个整数,表示Alice在哪个质因数上出错了。
%% this file is auto-generated. Do not edit it directly.


\subsection[样例 输入]{【样例 输入】}
\begin{minted}[linenos]{text}
1
2 3 1 10
\end{minted}

\subsection[样例 输出]{【样例 输出】}
\begin{minted}[linenos]{text}
2
\end{minted}


\subsection[样例 解释]{【样例 解释】}

将$10$质因数分解,得到$10=2\times 5$。接下来检查这两个质因数:$(2\times 3)\bmod 2=0\not=1$,$(2\times 3)\bmod 5=1$。因此,Alice在$2$这个质因数上出错了。





\subsection[数据范围]{【数据范围】}
对于所有测试数据,保证:$1\leq T\leq 1000$,$1\leq a,b,m,n\leq 10^{18}$。

每个测试点的具体限制见下表:
\begin{center}
  \begin{tabular}{c|c|c|c}
    \Xhline{5\arrayrulewidth}
    测试点编号      & $T\le $ & $a,b,m,n\leq$ & 特殊限制                \\
    \Xhline{3\arrayrulewidth}
    $1\sim2$   & $1000$  & $10^6$        &                     \\
    \hline
    $3\sim4$   & $1000$  & $10^{18}$     & $n$的所有质因数都$\le 100$ \\
    \hline
    $5\sim 6$  & $1$     & $10^{18}$     &                     \\
    \hline
    $7\sim 10$ & $1000$  & $10^{18}$     &                     \\
    \Xhline{5\arrayrulewidth}
  \end{tabular}
\end{center}

\clearpage

% problem statement for crt

\renewcommand{\cnname}{\protect\input{crt/1-CN-NAME}\unskip}
\renewcommand{\enname}{\protect\input{crt/2-EN-NAME}\unskip}

\section{\cnname(\englishname{\enname})}

\subsection[题目描述]{【题目描述】}
「物不知数」问题:
\begin{quote}
  有物不知其数,三三数之剩二,五五数之剩三,七七数之剩二。问物几何?
\end{quote}
即求满足以下条件的整数:除以 \(3\) 余 \(2\),除以 \(5\) 余 \(3\),除以
\(7\) 余 \(2\)。该问题最早见于《孙子算经》中,并有该问题的具体解法。宋朝数学家秦九韶于
1247
年《数书九章》卷一、二《大衍类》对「物不知数」问题做出了完整系统的解答。上面具体问题的解答口诀由明朝数学家程大位在《算法统宗》中给出:
\begin{quote}
  三人同行七十希,五树梅花廿一支,七子团圆正半月,除百零五便得知。
\end{quote}
\(2\times 70+3\times 21+2\times 15=233=2\times 105+23\),故答案为
\(23\)。

用现代数学的语言来说明的话,中国剩余定理 (Chinese Remainder Theorem, CRT)
可求解如下形式的一元线性同余方程组(其中 \(n_1, n_2, \cdots, n_k\)
两两互质):

\[
  \left\{  \begin{aligned}
    x & \equiv a_1 \pmod {n_1} \\
    x & \equiv a_2 \pmod {n_2} \\
      & \vdots                 \\
    x & \equiv a_k \pmod {n_k} \\
  \end{aligned}\right.
\]
上面的「物不知数」问题就是一元线性同余方程组的一个实例。
使用中国剩余定理求解一元线性同余方程组的步骤如下:
\begin{enumerate}
  \item
        计算所有模数的积 \(n\);
  \item
        对于第 \(i\) 个方程:

        \begin{enumerate}
          \item
                计算 \(m_i=\frac{n}{n_i}\);
          \item
                计算 \(m_i\) 在模 \(n_i\) 意义下的逆元
                \(m_i^{-1}\);
          \item
                计算 \(c_i=m_im_i^{-1}\)(不要对 \(n_i\) 取模)。
        \end{enumerate}
  \item
        方程组在模 \(n\) 意义下的唯一解为:\(x=\sum_{i=1}^k a_ic_i \pmod n\)。
\end{enumerate}
中国剩余定理的一个应用是,如果要计算$a\bmod n$,其中$n=p_1^{c_1}p_2^{c_2}\cdots p_k^{c_k}$,那么可以先计算$a_i=a\bmod p_i^{c_i}$,然后使用中国剩余定理求解出$a\mod n$的值,这样就可以在$n$是一个比较大的合数的时候减小乘法运算的规模。

Alice想要计算两个数字$a,b$的乘积对某个整数$n$取模的结果。在学习了中国剩余定理后,她决定采用上述的方法,先对$n$进行质因数分解,然后计算答案对每个质因数取模的结果,最后使用中国剩余定理求解出答案。但是很悲惨的是,在计算过程中,对于其中一个质因数的运算过程出现了错误,导致最终的结果也是错误的。虽然经过重新计算她得出了正确的结果,但是她还是想知道她之前的计算过程中在哪个质因数上出错了。
\subsection[输入格式]{【输入格式】}
从文件 \filename{\enname.in} 中读入数据。
输入的第一行包含一个整数 $T$,表示数据组数。

接下来 $T$ 行,每行包含四个整数 $a,b,m,n$,分别表示要计算乘积的两个数,Alice计算出的错误结果,以及模数。

\subsection[输出格式]{【输出格式】}
输出到文件 \filename{\enname.in} 中。
输出到文件 \filename{\enname.out} 中。

输出$T$行,每行一个整数,表示Alice在哪个质因数上出错了。
%% this file is auto-generated. Do not edit it directly.


\subsection[样例 输入]{【样例 输入】}
\begin{minted}[linenos]{text}
1
2 3 1 10
\end{minted}

\subsection[样例 输出]{【样例 输出】}
\begin{minted}[linenos]{text}
2
\end{minted}


\subsection[样例 解释]{【样例 解释】}

将$10$质因数分解,得到$10=2\times 5$。接下来检查这两个质因数:$(2\times 3)\bmod 2=0\not=1$,$(2\times 3)\bmod 5=1$。因此,Alice在$2$这个质因数上出错了。





\subsection[数据范围]{【数据范围】}
对于所有测试数据,保证:$1\leq T\leq 1000$,$1\leq a,b,m,n\leq 10^{18}$。

每个测试点的具体限制见下表:
\begin{center}
  \begin{tabular}{c|c|c|c}
    \Xhline{5\arrayrulewidth}
    测试点编号      & $T\le $ & $a,b,m,n\leq$ & 特殊限制                \\
    \Xhline{3\arrayrulewidth}
    $1\sim2$   & $1000$  & $10^6$        &                     \\
    \hline
    $3\sim4$   & $1000$  & $10^{18}$     & $n$的所有质因数都$\le 100$ \\
    \hline
    $5\sim 6$  & $1$     & $10^{18}$     &                     \\
    \hline
    $7\sim 10$ & $1000$  & $10^{18}$     &                     \\
    \Xhline{5\arrayrulewidth}
  \end{tabular}
\end{center}

\clearpage

% problem statement for crt

\renewcommand{\cnname}{\protect\input{crt/1-CN-NAME}\unskip}
\renewcommand{\enname}{\protect\input{crt/2-EN-NAME}\unskip}

\section{\cnname(\englishname{\enname})}

\subsection[题目描述]{【题目描述】}
「物不知数」问题:
\begin{quote}
  有物不知其数,三三数之剩二,五五数之剩三,七七数之剩二。问物几何?
\end{quote}
即求满足以下条件的整数:除以 \(3\) 余 \(2\),除以 \(5\) 余 \(3\),除以
\(7\) 余 \(2\)。该问题最早见于《孙子算经》中,并有该问题的具体解法。宋朝数学家秦九韶于
1247
年《数书九章》卷一、二《大衍类》对「物不知数」问题做出了完整系统的解答。上面具体问题的解答口诀由明朝数学家程大位在《算法统宗》中给出:
\begin{quote}
  三人同行七十希,五树梅花廿一支,七子团圆正半月,除百零五便得知。
\end{quote}
\(2\times 70+3\times 21+2\times 15=233=2\times 105+23\),故答案为
\(23\)。

用现代数学的语言来说明的话,中国剩余定理 (Chinese Remainder Theorem, CRT)
可求解如下形式的一元线性同余方程组(其中 \(n_1, n_2, \cdots, n_k\)
两两互质):

\[
  \left\{  \begin{aligned}
    x & \equiv a_1 \pmod {n_1} \\
    x & \equiv a_2 \pmod {n_2} \\
      & \vdots                 \\
    x & \equiv a_k \pmod {n_k} \\
  \end{aligned}\right.
\]
上面的「物不知数」问题就是一元线性同余方程组的一个实例。
使用中国剩余定理求解一元线性同余方程组的步骤如下:
\begin{enumerate}
  \item
        计算所有模数的积 \(n\);
  \item
        对于第 \(i\) 个方程:

        \begin{enumerate}
          \item
                计算 \(m_i=\frac{n}{n_i}\);
          \item
                计算 \(m_i\) 在模 \(n_i\) 意义下的逆元
                \(m_i^{-1}\);
          \item
                计算 \(c_i=m_im_i^{-1}\)(不要对 \(n_i\) 取模)。
        \end{enumerate}
  \item
        方程组在模 \(n\) 意义下的唯一解为:\(x=\sum_{i=1}^k a_ic_i \pmod n\)。
\end{enumerate}
中国剩余定理的一个应用是,如果要计算$a\bmod n$,其中$n=p_1^{c_1}p_2^{c_2}\cdots p_k^{c_k}$,那么可以先计算$a_i=a\bmod p_i^{c_i}$,然后使用中国剩余定理求解出$a\mod n$的值,这样就可以在$n$是一个比较大的合数的时候减小乘法运算的规模。

Alice想要计算两个数字$a,b$的乘积对某个整数$n$取模的结果。在学习了中国剩余定理后,她决定采用上述的方法,先对$n$进行质因数分解,然后计算答案对每个质因数取模的结果,最后使用中国剩余定理求解出答案。但是很悲惨的是,在计算过程中,对于其中一个质因数的运算过程出现了错误,导致最终的结果也是错误的。虽然经过重新计算她得出了正确的结果,但是她还是想知道她之前的计算过程中在哪个质因数上出错了。
\subsection[输入格式]{【输入格式】}
从文件 \filename{\enname.in} 中读入数据。
输入的第一行包含一个整数 $T$,表示数据组数。

接下来 $T$ 行,每行包含四个整数 $a,b,m,n$,分别表示要计算乘积的两个数,Alice计算出的错误结果,以及模数。

\subsection[输出格式]{【输出格式】}
输出到文件 \filename{\enname.in} 中。
输出到文件 \filename{\enname.out} 中。

输出$T$行,每行一个整数,表示Alice在哪个质因数上出错了。
%% this file is auto-generated. Do not edit it directly.


\subsection[样例 输入]{【样例 输入】}
\begin{minted}[linenos]{text}
1
2 3 1 10
\end{minted}

\subsection[样例 输出]{【样例 输出】}
\begin{minted}[linenos]{text}
2
\end{minted}


\subsection[样例 解释]{【样例 解释】}

将$10$质因数分解,得到$10=2\times 5$。接下来检查这两个质因数:$(2\times 3)\bmod 2=0\not=1$,$(2\times 3)\bmod 5=1$。因此,Alice在$2$这个质因数上出错了。





\subsection[数据范围]{【数据范围】}
对于所有测试数据,保证:$1\leq T\leq 1000$,$1\leq a,b,m,n\leq 10^{18}$。

每个测试点的具体限制见下表:
\begin{center}
  \begin{tabular}{c|c|c|c}
    \Xhline{5\arrayrulewidth}
    测试点编号      & $T\le $ & $a,b,m,n\leq$ & 特殊限制                \\
    \Xhline{3\arrayrulewidth}
    $1\sim2$   & $1000$  & $10^6$        &                     \\
    \hline
    $3\sim4$   & $1000$  & $10^{18}$     & $n$的所有质因数都$\le 100$ \\
    \hline
    $5\sim 6$  & $1$     & $10^{18}$     &                     \\
    \hline
    $7\sim 10$ & $1000$  & $10^{18}$     &                     \\
    \Xhline{5\arrayrulewidth}
  \end{tabular}
\end{center}

\clearpage





\end{document}