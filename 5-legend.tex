「物不知数」问题:
\begin{quote}
  有物不知其数,三三数之剩二,五五数之剩三,七七数之剩二。问物几何?
\end{quote}
即求满足以下条件的整数:除以 \(3\) 余 \(2\),除以 \(5\) 余 \(3\),除以
\(7\) 余 \(2\)。该问题最早见于《孙子算经》中,并有该问题的具体解法。宋朝数学家秦九韶于
1247
年《数书九章》卷一、二《大衍类》对「物不知数」问题做出了完整系统的解答。上面具体问题的解答口诀由明朝数学家程大位在《算法统宗》中给出:
\begin{quote}
  三人同行七十希,五树梅花廿一支,七子团圆正半月,除百零五便得知。
\end{quote}
\(2\times 70+3\times 21+2\times 15=233=2\times 105+23\),故答案为
\(23\)。

用现代数学的语言来说明的话,中国剩余定理 (Chinese Remainder Theorem, CRT)
可求解如下形式的一元线性同余方程组(其中 \(n_1, n_2, \cdots, n_k\)
两两互质):

\[
  \left\{  \begin{aligned}
    x & \equiv a_1 \pmod {n_1} \\
    x & \equiv a_2 \pmod {n_2} \\
      & \vdots                 \\
    x & \equiv a_k \pmod {n_k} \\
  \end{aligned}\right.
\]
上面的「物不知数」问题就是一元线性同余方程组的一个实例。
使用中国剩余定理求解一元线性同余方程组的步骤如下:
\begin{enumerate}
  \item
        计算所有模数的积 \(n\);
  \item
        对于第 \(i\) 个方程:

        \begin{enumerate}
          \item
                计算 \(m_i=\frac{n}{n_i}\);
          \item
                计算 \(m_i\) 在模 \(n_i\) 意义下的逆元
                \(m_i^{-1}\);
          \item
                计算 \(c_i=m_im_i^{-1}\)(不要对 \(n_i\) 取模)。
        \end{enumerate}
  \item
        方程组在模 \(n\) 意义下的唯一解为:\(x=\sum_{i=1}^k a_ic_i \pmod n\)。
\end{enumerate}
中国剩余定理的一个应用是,如果要计算$a\bmod n$,其中$n=p_1^{c_1}p_2^{c_2}\cdots p_k^{c_k}$,那么可以先计算$a_i=a\bmod p_i^{c_i}$,然后使用中国剩余定理求解出$a\mod n$的值,这样就可以在$n$是一个比较大的合数的时候减小乘法运算的规模。

Alice想要计算两个数字$a,b$的乘积对某个整数$n$取模的结果。在学习了中国剩余定理后,她决定采用上述的方法,先对$n$进行质因数分解,然后计算答案对每个质因数取模的结果,最后使用中国剩余定理求解出答案。但是很悲惨的是,在计算过程中,对于其中一个质因数的运算过程出现了错误,导致最终的结果也是错误的。虽然经过重新计算她得出了正确的结果,但是她还是想知道她之前的计算过程中在哪个质因数上出错了。