\documentclass{.latex-class/statement}

\title{{{title}}}

\fancyhead[R]{\footnotesize \MakeLowercase{\fancyplain{}{\leftmark}}}
\renewcommand{\sectionmark}[1]{\markboth{#1}{}} 


\begin{document}
\begin{titlepage}
  \vspace*{-20mm}
  \begin{center}
    \LARGE \sffamily \thetitle \\
    {{subtitle}}\\
    % \par\mdseries\LARGE\SubtitleContent\vskip 0.5em

    \large \rmfamily 时间:{{date}} {{start}} \textasciitilde\ {{end}}
    \vskip 0.5em
  \end{center}

  \begin{center}
    {{meta}}
  \end{center}

  提交源程序文件名
  \begin{center}
    {{fns}}
  \end{center}

  编译选项
  \begin{center}
    \begin{tabularx}{\textwidth}{|p{3.2cm}|X|X|X|}
      \hline
      对于 C++ & \multicolumn{3}{>{\centering\hsize=\dimexpr3\hsize+4\tabcolsep+2\arrayrulewidth\relax}X|}{\texttt{-O2 -std=c++14 -static}} \\
      \hline
    \end{tabularx}\par
  \end{center}

  \stress{注意事项(请仔细阅读)}
  \begin{enumerate}
    \item 文件名(程序名和输入输出文件名)必须使用英文小写。
    \item C/C++ 中函数 main() 的返回值类型必须是 int,程序正常结束时的返回值必须是 0。
    \item 提交的程序代码文件的放置位置请参考各省的具体要求。
    \item 因违反以上三点而出现的错误或问题,申诉时一律不予受理。
    \item 若无特殊说明,结果的比较方式为全文比较(过滤行末空格及文末回车)。
    \item 选手提交的程序源文件必须不大于 100KB。
    \item 程序可使用的栈空间内存限制与题目的内存限制一致。
    \item 全国统一评测时采用的机器配置为:Intel(R) Core(TM) i7-8700K CPU @3.70GHz,
          内存 32GB。上述时限以此配置为准。
    \item 只提供 Linux 格式附加样例文件。
    \item 评测在当前最新公布的 NOI Linux 下进行,各语言的编译器版本以此为准。
  \end{enumerate}
\end{titlepage}

\clearpage

{{prob-statements}}

\end{document}