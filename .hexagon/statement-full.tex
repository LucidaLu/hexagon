\documentclass{\string~/.hexagon/assets/statement}

\title{{{title}}}

\fancyhead[R]{\footnotesize \MakeLowercase{\fancyplain{}{\leftmark}}}
\renewcommand{\sectionmark}[1]{\markboth{#1}{}} 


\begin{document}
\begin{titlepage}
  \vspace*{-20mm}
  \begin{center}
    \LARGE \sffamily \thetitle \\
    {{subtitle}}\\
    % \par\mdseries\LARGE\SubtitleContent\vskip 0.5em

    \large \rmfamily 时间:{{date}} {{start}} \textasciitilde\ {{end}}
    \vskip 0.5em
  \end{center}

  \begin{center}
    {{meta}}
  \end{center}

  提交源程序文件名
  \begin{center}
    {{fns}}
  \end{center}

  编译选项
  \begin{center}
    \begin{tabularx}{\textwidth}{|p{3.2cm}|X|X|X|}
      \hline
      对于 C++ & \multicolumn{3}{>{\centering\hsize=\dimexpr3\hsize+4\tabcolsep+2\arrayrulewidth\relax}X|}{\texttt{-O2 -std=c++14 -static}} \\
      \hline
    \end{tabularx}\par
  \end{center}

  \begingroup\titleformat{\subsection}{\bf}{}{0pt}{\hspace{0.5em}}\subsection{注意事项与提醒(请选手务必仔细阅读)}\endgroup
  \begin{enumerate}
    \item 选手提交的源程序必须存放在\stress{已建立}好的,且\stress{带有样例文件和下发文件的}的文件夹中,文件名称与对应试题英文名一致;
    \item 文件名(包括程序名和输入输出文件名)必须使用英文小写。
    \item C++ 中函数 main() 的返回值类型必须是 int,值必须为 0。
    \item \stress{对于因未遵守以上规则对成绩造成的影响,相关申诉不予受理}。
    \item 若无特殊说明,结果比较方式为\stress{忽略行末空格、文末回车后的全文比较。}。
    \item 程序可使用的栈空间大小与该题内存空间限制一致。
    \item 在终端中执行命令 \texttt{ulimit -s unlimited} 可将当前终端下的栈空间限制放大,但你使用的栈空间大小不应超过题目限制。
    \item 若无特殊说明,每道题的\stress{代码大小限制为 100KB}。
    \item 若无特殊说明,输入与输出中同一行的相邻整数、字符串等均使用一个空格分隔。
    \item 输入文件中可能存在行末空格,请选手使用更完善的读入方式(例如 scanf 函数)避免出错。
    \item 直接复制 PDF 题面中的多行样例,数据将带有行号,建议选手直接使用对应目录下的样例文件进行测试。
    \item 使用 std::deque 等 STL 容器时,请注意其内存空间消耗。
    \item 请务必使用题面中规定的的编译参数,保证你的程序在本机能够通过编译。此外\stress{不允许在程序中手动开启其他编译选项},一经发现,本题成绩以 0 分处理。
  \end{enumerate}
\end{titlepage}

\clearpage

{{prob-statements}}

\end{document}